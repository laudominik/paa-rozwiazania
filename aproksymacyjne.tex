\documentclass{article}

\title{Algorytmy aproksymacyjne}
\author{Dominik Lau}

\usepackage{blindtext}
\usepackage{amsmath}
\usepackage[utf8]{inputenc}
\usepackage[polish]{babel}
\usepackage[T1]{fontenc}
\usepackage{listings}
\usepackage{color}
\usepackage{amssymb}
\usepackage{esvect}
\usepackage{graphicx}


\graphicspath{ {./obrazy/} }

\definecolor{dkgreen}{rgb}{0,0.6,0}
\definecolor{gray}{rgb}{0.5,0.5,0.5}
\definecolor{mauve}{rgb}{0.58,0,0.82}

\lstset{frame=tb,
  language=Python,
  aboveskip=3mm,
  belowskip=3mm,
  showstringspaces=false,
  columns=flexible,
  basicstyle={\small\ttfamily},
  numbers=none,
  numberstyle=\tiny\color{gray},
  keywordstyle=\color{blue},
  commentstyle=\color{dkgreen},
  stringstyle=\color{mauve},
  breaklines=true,
  breakatwhitespace=true,
  tabsize=3
}


\begin{document}

\maketitle

\section{Definicje}
\subsection{Algorytm k-absolutnie aproksymacyjny}
Jest to taki algorytm A, który dla pewnych danych $I$ gwarantuje
\begin{gather*}
	|A(I) - OPT(I)| \leq k
\end{gather*}
\subsection{Algorytm k(względnie)-aproksymacyjny}
Jest to taki algorytm A, który dla pewnych danych $I$ gwarantuje dla problemu minimalizacyjnego
\begin{gather*}
	\frac{A(I)}{OPT(I)} \leq k
\end{gather*}
a dla maksymalizacyjnego
\begin{gather*}
	\frac{OPT(I)}{A(I)} \leq k
\end{gather*}

\subsection{Schemat aproksymacyjny}
dla problemu minimalizacyjnego algorytm A jest schematem aproksymacyjnym
jeśli dla każdego $\varepsilon > 0$ zachodzi
\begin{gather*}
	\frac{A(I)}{OPT(I)} \leq 1 + \varepsilon
\end{gather*}
\textbf{po ludzku}: mamy algorytm aproksymacyjny,  którego dokładność możemy regulować (tylko, że czym większa dokładność, tym większa złożoność algorytmu)\\\\
\textbf{PTAS} - (czyt. "pitas") schemat wielomianowy względem rozmiaru problemu,  np.  $O(n^{\frac{1}{\varepsilon}})$ \\
\textbf{EPTAS} - schemat wielomianowy względem n, w którym $\varepsilon$ nie jest w potędze n \\
\textbf{FPTAS} - schemat wielomianowy względem rozmiaru  i $\frac{1}{\varepsilon}$, np. $O((\frac{1}{\varepsilon})^2n)$

\subsection{Problemy silnie NP-trudne}
Jest to problem, dla którego istnieje jego podproblem,  który jest również NP-trudny.
Dla takich aglorytmów nie ma FPTAS.

\section{Przykłady algorytmów aproksymacyjnych}

\textbf{NaiveColor} - przybliżone kolorowanie krawędzi
\begin{lstlisting}
def NaiveColor(G):
	c = 1
	for e in E(G):
		C = set{1..c}
		C.usun_nielegalne_kolory(e)

		if C.empty():
			c+=1
			color(e) = c
		else:	
			color(e) = pick_any(C)
\end{lstlisting}
dla każdej krawędzi przeglądamy maksymalnie $2\Delta$ kolorów innych krawędzi (a z tw.Vizinga $\Delta \leq \chi' \leq \Delta + 1$), więc pomylimy się góra dwukrotnie zatem jest to algorytm 2-aproksymacyjny \\\\
\textbf{NaiveCover} - przybliżone pokrycie wierzchołkowe
\begin{lstlisting}
def NaiveCover(G):
	C = set()
	while not E.empty():
		{u,v} = pick_any(E)		
		C.add(u) # !!!
		C.add(v) # !!!
		E.remove_incident(u)
		E.remove_incident(v)
	return C
\end{lstlisting}
dla wybranej krawędzi umieszczamy w pokryciu oba wierzchołki (zamiast jednego), więc w najgorszym przypadku umieścimy w pokryciu dwa razy za dużo wierzchołków - algorytm jest 2-aproksymacyjny\\\\
\textbf{kolorowanie krawędzi z tw. Vizinga}
\begin{lstlisting}
def ChromaticIndex(G):
	return Delta(G)
\end{lstlisting}
jest to algorytm 1-absolutnie aproksymacyjny (bo czasem się pomylimy o jeden kolor) oraz $\frac{4}{3}$-aproksymacyjny \\\\
\textbf{kolorowanie wierzchołkowe grafów planarnych}
\begin{lstlisting}
def ChromaticNumberPlanar(G):
	if empty(G): return 1
	if bipartite(G): return 2
	return 4
\end{lstlisting}
jest to algorytm 1-absolutnie aproksymacyjny - czasem mylimy się o jeden kolor (jeśli $\chi = 3$) oraz $\frac{4}{3}$-aproksymacyjny \\\\
\textbf{podstawowy algorytm do rozwiązywania problemu komiwojażera}

\begin{lstlisting}
def TSP_approx(G):
	MST = findMinimumSpanningTree(G)
	path = connect_vertices_in_triangles(G)
	return path
\end{lstlisting}
jest to algorytm 2-aproksymacyjny na płaszczyźnie \\\\
\textbf{przykłady FPTAS} \\
Problem plecakowy: $O(n(logn + \frac{1}{\varepsilon^2}))$ oraz $O(nlog(\frac{1}{\varepsilon}) + \frac{1}{\varepsilon^4} )$ \\
Problem sumy podzbioru: $O(n + \frac{1}{\varepsilon^3})$ \\\\




\end{document}
